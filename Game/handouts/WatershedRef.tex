\documentclass[10pt,twocolumn]{article}

\usepackage[no-math]{fontspec}
\defaultfontfeatures{Scale=MatchLowercase,Mapping=tex-text}
\setmainfont{Times New Roman}
\setsansfont{Courier}
\usepackage{balance}

\usepackage[margin=.5in]{geometry}
\setlength{\columnsep}{1cm}
\begin{document}

\thispagestyle{empty}

\twocolumn[
\centerline{\LARGE \bf  Reference Sheet}

\medskip
\hrule
\medskip

]


\noindent\textbf{\large Watershed}
\begin{description}

\item{}
Watershed contains all the Tiles of the game map.

\item{\texttt{WS.getAllTiles()}}\ \\[.25em]
%
  Returns an array containing all the Tiles in the game map.

\item{\texttt{WS.update()}}\ \\[.25em]
%
  Updates all the land uses of the game map to have the current pollution values.
  Also computes and updates the total pollution in the watershed using current pollution values.

\item{\texttt{WS.totalDecayPollution}}\ \\[.25em]
%
  The total amount of pollution that reaches the river after distance decay.

\end{description}

\noindent\textbf{\large Tile}
\begin{description}

\item{}
A Tile is one square on the game map. It contains the tiles position on the game grid and its land use. 

\item{\texttt{t.getX()}}\ \\[.25em]
%
  Returns the $x$ coordinate of \texttt{t} on the canvas. 

\item{\texttt{t.getY()}}\ \\[.25em]
%
  Returns the $y$ coordinate of \texttt{t} on the canvas. 

item{\texttt{t.getDecayPollution()}}\ \\[.25em]
%
  Returns the pollution that reaches the river after distance decay of \texttt{t}.

\end{description}

\noindent\textbf{\large Landuse}
\begin{description}

\item{}
Each tile on the game map has one of six landuses: Factory, Farm, House, Forest, Dirt or River.

\item{\texttt{lu.getType()}}\ \\[.25em]
%
  Returns the land use of \texttt{lu}. There are six possible values:
  \begin{enumerate}
  \item\texttt{LUType.FACTORY}
  \item\texttt{LUType.FARM}
  \item\texttt{LUType.HOUSE}
  \item\texttt{LUType.FOREST}
  \item\texttt{LUType.DIRT}
  \item\texttt{LUType.RIVER}
  \end{enumerate}

\end{description}


\medskip

\noindent\textbf{\large Conditionals and Return}
\begin{description}

\item{\texttt{if(b) \{ ... \} else \{ ... \}}}\ \\[.25em] 
% 
Conditionally executes the first block of code if \texttt{b} evaluates to
\texttt{true}, and the second block of code if it evaluates to \texttt{false}.

\item{\texttt{if( (a)\&\&(b) ) \{ ... \} else \{ ... \}}}\ \\[.25em] 
% 
Conditionally executes the first block of code if both \texttt{a} and \texttt{b} evaluates to
\texttt{true}, and the second block of code if either evaluates to \texttt{false}.

\item{\texttt{return b}} \ \\[.25em]
%
  Returns \texttt{b} as the value of the current method.
\end{description}
\medskip

\medskip

\balance

\noindent\textbf{\large Assignment and Arithmetic}
\begin{description}

\item{\texttt{x = n}} \ \\[.25em]
  Assigns variable \texttt{x} to the value \texttt{v}.

\item{\texttt{n1 + n2}} \ \\[.25em]
  Adds \texttt{n1} to \texttt{n2}. 

\item{\texttt{n1 - n2}} \ \\[.25em]
  Subtracts \texttt{n1} from \texttt{n2}. 

\item{\texttt{n1 * n2}} \ \\[.25em]
  Multiplies \texttt{n1} by \texttt{n2}. 

\item{\texttt{n1 / n2}} \ \\[.25em]
  Divides \texttt{n1} by \texttt{n2} (with no remainder).
  
 \item{\texttt{n1 > n2}} \ \\[.25em]
Returns \texttt{true} if \texttt{n1} is strictly greater than \texttt{n2}. Otherwise returns \texttt{false}.

 \item{\texttt{n1 < n2}} \ \\[.25em]
Returns \texttt{true} if \texttt{n1} is strictly less than \texttt{n2}.Otherwise returns \texttt{false}.
\end{description}

\medskip

\noindent\textbf{\large Arrays}

\begin{description}
\item{\texttt{new A[n]}} \\[.25em]
Creates an new array objects of type \texttt{A} of length
\texttt{n}. 

\item{\texttt{A[] = \{ v1, ... vk \};}} \\[.25em] 
Creates a new array initialized with elements \texttt{v1} through \texttt{vk}. 

\item{\texttt{a.length}} \\[.25em]
The length of array \texttt{a}. 

\item{\texttt{a[i]}} \\[.25em]
The \texttt{i}\textsuperscript{th} element of array \texttt{a}. 

\item{\texttt{a[i] = v}} \\[.25em]
Stores \texttt{v} as the \texttt{i}\textsuperscript{th} element of array \texttt{a}. 

\item{\texttt{for(int i = 0; i < a.length; i++) \{ ... \}}} \\[.25em]
  Iterates through the elements of array \texttt{a}. In each iteration
  of the loop, the element can be accessed using index \texttt{i}.
  
\end{description}

\noindent\textbf{\large Files}
\begin{description}
\item{\texttt{Exercise}}\ \\[.25em]
Your code goes here.

\item{\texttt{Game}}\ \\[.25em] 
Defines the Watershed data structure and all gameplay functionality.

\item{\texttt{GUI}}\ \\[.25em] 
The Graphical User Interface. This defines the main \texttt{draw} method, which is invoked every time the screen refreshes, and is responsible for everything you see in the frame.

\item{\texttt{LandUse}}\ \\[.25em] 
Defines the \texttt{LandUse} data structure.

\item{\texttt{Tile}}\ \\[.25em] 
Defines the \texttt{Tile} data structure.

\end{description}

\medskip

\noindent\textbf{Advanced Graphics}
\begin{description}
\item{\texttt{background(c)}}\ \\[.25em]
%
Sets the background color to \texttt{c}. There are many ways to describe colors: 
\begin{itemize}
\item \texttt{n}: a single number between \texttt{0} (black) and \texttt{255} (white) describes a shade of grey. 
\item \texttt{(r,g,b)}: a triple of numbers each ranging from
  \texttt{0} to \texttt{255}, describe a color in terms of the
  relative amounts of red, green, and blue.
\end{itemize}

\item{\texttt{fill(c)}}\ \\[.25em]
Sets the current fill color to \texttt{c}. 

\item{\texttt{ellipse(x,y,w,h)}}\ \\[.25em]
Draws an ellipse at coordinates \texttt{x} and \texttt{y} with witdth \texttt{w} and height \texttt{h}. 

\item{\texttt{rect(x,y,w,h)}}\ \\[.25em]
Draws a rectangle at coordinates \texttt{x} and \texttt{y} with witdth \texttt{w} and height \texttt{h}. 

\item{\texttt{mouseX}}\ \\[.25em]
%
The current $x$ coordinate of the mouse pointer.

\item{\texttt{mouseY}}\ \\[.25em]
%
The current $y$ coordinate of the mouse pointer.

\item{\texttt{mouseClicked()}}\ \\[.25em]
Invoked each time the mouse button is clicked.

\item{\texttt{keyPressed()}}\ \\[.25em] 
Invoked each time any key is pressed. The key pressed can be accessed
from \texttt{key}.
\end{description}

\end{document}
